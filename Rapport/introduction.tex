\subsection{Enoncé}
Le département ASI ne dispose pas encore de compte Twitter (ou je ne l'ai pas trouvé !).
 
La première chose à faire est de créer ce compte. Le but de ce compte est de mieux faire connaitre le département ASI auprès des 
ses futurs élèves ingénieurs. Il s'agit de cibler des étudiants de première, terminale et IUT susceptibles de venir en ASI mais qui 
n'en n'ont pas encore l'idée aujourd'hui. 
 
Le deuxième chose à faire est donc de définir précisément le public visé et ses attentes pour qu'à la fois il aie envie de s'abonner 
et, idéalement, de venir faire ses études chez nous.
 
La troisième chose à faire est de recenser des contenus susceptibles de nous permettre d'atteindre nos objectifs, et de les partager via 
le compte nouvellement créé. 

\subsection{Approche adoptée}
Afin de remplir les différents objectifs du sujet, nous allons suivre différentes étapes. Tout d'abord, par le biais de ce dossier, nous allons 
étudier le marché de la communication sur Internet : quels sont les moyens utilisés, les contenus publiés, les publics visés sur la toile 
pour gérer une communauté d'internautes.

Ensuite, une fois familiers avec ce milieu, il nous faudra mettre en place une stratégie adaptée à notre cas. Allons-nous n'utiliser que Twitter, ou 
nous faudra-t-il voir plus large pour garantir un succès ?

Enfin arrivera la partie pratique, où il nous faudra mettre en place tout ce que nous aurons appris et décidé en amont.
