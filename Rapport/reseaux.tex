\subsection{La révolution des réseaux sociaux}
La communication a été théorisé par Claude Shannon en 1948 et a été portée par la suite aux sciences humaines.

Pour transmettre un message, il est nécessaire d'avoir deux entités : l'émetteur et le récepteur. Actuellement, le département prend le rôle d'émetteur tandis que les élèves et étudiants recherchant une formation prennent eux le rôle de récepteur. Ce mode de transmission de l'information est souvent utilisé par le département ASI et l'INSA de Rouen, par exemple en distribuant des dépliants publicitaires.

\begin{center}
\includegraphics[scale=0.5]{./image/information.png}
\end{center}

L'étape suivant l'information est la communication. Le processus est similaire, mais cette fois, le récepteur renvois un message vers l'émetteur : une réaction, une question. C'est ce retour qui valide si le processus de communication est efficace. La communication est notamment utilisée lors des salons, des présentations, quand des ambassadeurs de l'INSA se déplace vers le publique destinataire.

\begin{center}
\includegraphics[scale=0.5]{./image/communication.png}
\end{center}

Avec l'arrivée des réseaux sociaux, la tendance s'est inversée. Ce n'est plus l'émetteur original mais le récepteur qui envois le message. Sur la page, mur, blog, fil d'actualité, ce n'est plus l'émetteur qui envois des messages, mais sa communauté. L'émetteur se doit alors de répondre, transmettre l'information, les messages et commentaire de sa communauté.

\begin{center}
\includegraphics[scale=0.5]{./image/communication_digitale.png}
\end{center}

\subsection{Le métier de community manager}

Du fait de l'essor de ces nouveaux réseaux sociaux, de nouveaux métiers se sont également développés, très liés à ces médias et au web 2.0.

\subsection{Présence du public ciblé sur les réseaux sociaux}
