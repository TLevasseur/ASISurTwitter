\subsection{ASI sur les réseaux sociaux}
Les informations qu'il est possible de partager sur les réseaux sociaux sont nombreuses et variées. Dans tous les cas, le but est de 
créer une réaction, afin que l'information soit partagée au maximum. Dans notre cas, les informations suivantes semblent appropriées :
\begin{itemize}
	\item informations insolites (robot puissance4),
	\item astuces utiles (offre de telle ou telle compagnie pour les étudiants),
	\item information de fonctionnement (rentrée, choix des PIC),
	\item retransmission d'information sur le département ou l'INSA (article de journal),
	\item mise en avant de personnalité (gagnant de concours, personnel du département),
	\item information sur l'actualité relative au domaine ASI.
\end{itemize}


\subsection{ASI sur twitter}
L'objectif de notre PAO est bien de créer et faire briller le compte Twitter du département, Il apparaît qu'il est tout à fait possible 
de mettre en place relativement rapidement un compte Twitter acitf.

Dans un premier temps, il sera nécessaire de s'approprier la communauté déjà présente et intéressée par le sujet en apparaissant sur les 
pages de comptes actifs existant tels que ceux des six INSA, celui de NWX ou des revues spécialisés par exemple.

Il est indispensable de sortir de l'anonymat et d'être vu.

\subsection{Mais aussi sur Facebook}
Sur Facebook, la mise en place est légèrement différente, au sens ou être partagé par d'autres déjà bien implantés n'apporte pas une 
visibilité conséquente. La seule solution viable est donc de grossir progressivement par le bouche à oreille. Néanmoins, même si le travail 
semble de beaucoup plus longue haleine que celui à fournir sur Twitter, le résultat sera aussi beaucoup plus fort, au vu du taux de 
pénétration et du temps passé par les adolescents sur Facebook.
