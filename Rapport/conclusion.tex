\paragraph{}
Ainsi, ce dossier nous montre comment le public français, et notamment les jeunes, réagit face aux différents réseaux sociaux, sur lesquels toutes sortent 
d'entités bombardent de contenu, images, vidéos, etc. Comme nous avons pu le voir, ils sont réceptifs à certains genres de contenu, et c'est à ceux-ci que 
nous devons nous intéresser afin de réussir à augmenter la visibilité de notre département sur Internet.

\paragraph{}
Nous savons donc maintenant quelles astuces, quelles stratégies mettre en place pour utiliser ces nouveaux médias avec la meilleure efficacité possible, et 
sommes prêts à nous initier au métier de gestionnaire de communauté de la façon la plus concrète : en devenir nous-mêmes le temps de ce projet.

\paragraph{}
Pour cela, il nous faut non seulement utiliser Twitter, comme le sujet nous le demande, mais également Facebook, qui fait la paire avec le premier. Grâce à 
l'association de ces deux outils et leur utilisation refléchie, il nous semble possible de dynamiser l'activité du département ASI et son attrait pour 
les différentes catégories d'étudiants. 
