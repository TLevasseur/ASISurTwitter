\section{Évolutions futures}
\subsection{Rendez-vous hebdomadaire}
	Un bon moyen de fidéliser l'utilisateur serait de créer un véritable rendez-vous hebdomadaire qui permettrait de mettre en avant le département. Une idée d'interview des anciens ASI serait tout à fait envisageable. Ce rendez-vous permettrait aux potentiels futurs élèves de voir quels sont les carrières post-ASI possibles, et donnerait possiblement des contacts ou des idées de stages pour les élèves actuels. Ce travail est donc à coupler avec une recherche de contacts des anciens élèves ASI, qui pourrait se faire, justement, en utilisant les réseaux sociaux : les élèves d'une promotion connaissent souvent ceux de la promotion précédente grâce à l'intégration et au système de parrainage, il doit donc être possible de remonter, génération par génération, l'arbre des étudiants ASI.
	
\subsection{Affirmer la présence ASI lors des forums étudiants}
	Maintenant que le département s'est doté de pages sur les réseaux sociaux, il est important d'être vu et de donner à voir. C'est en étant au plus près des élèves, en leur distribuant des tracts, en leur montrant que l'on existe, que le département gagnera à être connu et accélèrera le processus de recrutement. L'INSA organise très souvent des forums étudiants ou des interventions dans des établissements scolaires pour communiquer auprès des potentiels futurs élèves : il serait plus qu'intéressant de motiver les étudiants à s'y rendre pour vanter les mérites de notre département. L'affiche comprenant les deux QRCodes pourrait être mise à disposition des étudiants se rendant dans ces différents forums afin donner une visibilité au département.
	
\subsection{Lancer des débats}
	Sans entrer dans la polémique, participer à des débats de société, comme par exemple la place du libre en informatique où les futurs amendements à la loi informatique et liberté. Le but n'étant pas de positionner le département en temps que militant d'un bord ou d'un autre, mais de mettre en place sur ses pages une zone de débats et d'échanges sur le sujet.
