\section{Évolutions futures}
\subsection{Rendez-vous hebdomadaire}
	Un bon moyen de fidéliser l'utilisateur serait de créer un véritable rendez-vous hebdomadaire qui permettrait de mettre en avant le département. 
	Une idée d'interview des anciens ASI serait tout à fait envisageable. Ce rendez-vous permettrait aux potentiels futurs élèves de voir quels sont les carrières post-ASI possibles, 
	et donnerait possiblement des contacts ou des idées de stages pour les élèves actuels.
	
\subsection{Affirmer la présence ASI lors des forums étudiants}
	Maintenant que le département s'est doté de pages sur les réseaux sociaux, il est important d'être vu et de donner à voir. C'est en étant au plus près des élèves, en leur distribuant 
	des tracts, en leur montrant que l'on existe que le département gagnera à être connu et accélèrera le processus de recrutement. L'affiche comprenant les deux QRCodes pourra être mise 
	à disposition des étudiants se rendant dans ces différents forums afin donner une visibilité au département.
	
\subsection{Lancer des débats}
	Sans entrer dans la polémique, participer à des débats de société, comme par exemple la place du libre en informatique où les futurs amendements à la loi informatique et liberté. Le 
	but n'étant pas de positionner le département en temps que militant d'un bord ou d'un autre, mais de mettre en place sur ses pages une zone de débats et d'échanges sur le sujet.
