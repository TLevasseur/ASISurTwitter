\section{Retour sur expérience}
\subsection{Thibaud : Conclusion}

\paragraph{}
Ce PAO m'a réellement ouvert sur les réseaux sociaux et sur la portée que peuvent avoir certains messages ou publications, notamment grâce à la publication des commémorations des attentats. J'ai compris qu'il fallait 
accorder un soin particulier à ce que l'on poste sur internet, surtout durant cette période de tensions. En revanche, il est aussi possible d'utiliser les réseaux sociaux de façon habile et de jouer pour mettre en 
avant ce qui peut servir à un plan com.
Néanmoins, nous n'inventons rien et ne faisons que copier nos aînés comme par exemple Louis XIV qui se faisant peindre dans son manteau d'hermine avec sa tête d'homme mature pour évoquer la sagesse et des jambes 
de jeunes danseur connotant la vigueur.
Les réseaux sociaux sont des outils utilisable par toute entité physique ou morale qu'il est nécessaire de maîtriser. La métaphore de l'arme à double tranchant prend ici tout son sens.

\subsection{Quentin : Conclusion}

\paragraph{}
Dès l'annonce des sujets de PAO au début du semestre, mon choix s'est porté sur celui-ci. En effet, ayant été responsable communication du Bureau Des Elèves de l'INSA de Rouen l'année dernière, ce sujet m'a paru 
l'occasion rêvée de mettre en pratique ce que j'ai pu apprendre en assumant ce rôle, tout en le faisant dans un cadre plus formel, moins tourné vers la vie étudiante en dehors des cours.

\paragraph{}
J'ai découvert ce qu'était réellement la plateforme Twitter, sur laquelle je n'avais que des préjugés, et ai réalisé l'usage puissant qu'on pouvait en faire. Je me suis aussi rendu compte qu'il est difficile de 
créer une communauté : dans le cadre du BDE, la page Facebook était déjà active et vivante avant que je prenne mes fonctions. Ici, nous avons du partir de zéro, tout était à faire.

\paragraph{}
La communication et les réseaux sociaux sont donc des outils puissants mais délicats à bien utiliser : il faut savoir créer, et surtout fidéliser la communauté.
