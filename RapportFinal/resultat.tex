\section{Résultats}

\subsection{Twitter}

Le compte twitter totalise un total de 50 abonnés (20/01/2015), en augmentation lente mais constante. On compte parmi ceux-ci des élèves et des anciens élèves, des écoles et des comptes issu du monde universitaire, ainsi que des concours et des professionnels du milieu informatique.


\subsubsection{Ce qui a marché}
\paragraph{Les PICs ASI adoubés une nouvelle fois par norme ISO9001.}
\subsubsection{Ce qui n'a pas marché}

\subsection{Facebook}

La page Facebook totalise un total de 104 abonnés (20/01/2015). La page a connu un énorme pic d'abonnement lors de sa création, essentiellement par les élèves du département, mais plus généralement par les élèves de l'INSA de Rouen. On compte aussi d'ancien élèves de l'INSA, ainsi que leurs proches. Les publications ont atteins au maximum 321 personnes, dont 228 non abonnés.

Les deux graphiques suivant montre un corrélation visuelle imédiate entre la portée de la publication et l'engagement que celle-ci à suscité (mentions "j'aime", partages et commentaires). Il est donc nécessaire de prendre le soin de provoquer cet engagement. 
\begin{center}
\includegraphics[width=0.90\textwidth]{images/PorteeDesPublication.png}
\end{center}


\begin{center}
\includegraphics[width=0.90\textwidth]{images/Engagements.png}
\end{center}

La tableau ci-dessous montre les résultats des 10 dernières publications faites sur la page ASI et met en évidence pour chaque publication les résultats de celles-ci.

\begin{center}
	\includegraphics[width=0.90\textwidth]{images/DernierePublication.png}
\end{center}

\subsubsection{Ce qui a marché}
\paragraph{La pire nouvelle du dimanche soir}
Cet publication est une publication simple à but humoristique contenant une phrase courte et incomplète se référent à la photo publiée avec. Elle s'adresse à tous les étudiants qui ont souvent cet angoisse du lendemain de savoir à quelle heure commence la journée du lendemain.
\begin{center}
	\includegraphics[width=0.90\textwidth]{images/laPireNouvelle.png}
\end{center}

\paragraph{Ca sent le PAO à plein nez !}
Le fameux logiciel de rencontre Tinder permet de juger directement.


\paragraph{Pour ceux qui avaient prévu de piquer les derniers joujous high-tech des ASI4 }
Nous avons choisi de montrer, avec l'accord de M. Serge Dubois, les vidéos de l'évaluation de combat afin de montrer que même si nous sommes la plupart du temps devant des ordinateurs, nous ne sommes pas pour autant inactif !
Les différentes vidéos de Judo ont suscité un intérêt notable auprès des abonnés de la page Facebook. L'ensemble des vidéos totalise plus de 150 visionnages unitaires et a atteint plus de 300 personnes. C'est certainement le plus gros succès de la page Facebook.


\subsubsection{Ce qui n'a pas marché}


\subsection{Rayonnement particulier}

Lors des tragiques évènement de début décembre, une minute de silence a été organisé à l'INSA devant les drapeaux près du parking Magellan. Nous avons décidé de couvrir l'événement en partageant le communiqué des présidents d'université ainsi qu'une photo des commémorations le jeudi 8 janvier. 
Le lundi suivant, le cliché est en tête du carrousel des photo des commémorations sur le site http://www.normandie-actu.fr/, crédité à la page Facebook du département ASI de l'INSA de Rouen.
Bien que ce partage soit dû au contexte exceptionnel de la situation, il montre cependant que la presse locale consulte notre page Facebook et/ou notre compte Twitter.