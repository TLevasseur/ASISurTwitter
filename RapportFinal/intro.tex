\section{Introduction}

\subsection{Rappel de l'énoncé}
Le département ASI ne dispose pas encore de compte Twitter (ou je ne l'ai pas trouvé !).
 
La première chose à faire est de créer ce compte. Le but de ce compte est de mieux faire connaitre le département ASI auprès des 
ses futurs élèves ingénieurs. Il s'agit de cibler des étudiants de première, terminale et IUT susceptibles de venir en ASI mais qui 
n'en n'ont pas encore l'idée aujourd'hui. 
 
Le deuxième chose à faire est donc de définir précisément le public visé et ses attentes pour qu'à la fois il aie envie de s'abonner 
et, idéalement, de venir faire ses études chez nous.
 
La troisième chose à faire est de recenser des contenus susceptibles de nous permettre d'atteindre nos objectifs, et de les partager via 
le compte nouvellement créé. 

\subsection{Objectifs}

\paragraph{}
Les objectifs tels qu'ils étaient décrits dans l'énoncé du PAO sont multiples. Dans un premier temps, il était nécessaire de créer et gérer le compte Twitter du département ASI auquel nous avons couplé une page Facebook. Dans une seconde partie, il a fallu rendre ces comptes dynamiques et attrayants afin de réussir à rassembler une communauté autour de département, particulièrement de jeunes lycées ou étudiants potentiellement intéressés par la formation proposée par ASI.

\paragraph{}
Nous avons donc, pendant ce semestre, endossé différents rôles. Celui, tout d'abord, de community manager, ayant pour but de dynamiser les communautés autour des réseaux sociaux. Ensuite, il a fallu, via ces comptes, rejoints des communautés du monde de l'informatique afin de pouvoir récolter, et potentiellement partager, un maximum d'informations intéressantes à nos abonnés. Enfin, nous nous sommes essayé au monde du journalisme pour récolter nous-mêmes des informations sur le terrain.