\section{Conclusion}

\paragraph{}
Ce premier semestre de présence du département ASI sur les réseaux sociaux s'est donc révélé plutôt fructueux. Avec 100 mentions "J'aime" sur Facebook et une cinquantaine de followers sur Twitter, la communauté autour de nous s'est bien créée et grandit lentement mais sûrement. Nous espérons très fortement que le PAO sera repris par des étudiants d'ASI 3 ou 4 : il serait fort dommage que les efforts que nous avons founis aient été vains et que les deux comptes tombent dans l'oubli des tréfonds d'internet...

\paragraph{}
Nos successeurs ne seront pas en reste. Il faudra tout d'abord tout faire pour éviter une période de vide au moment de la passation entre notre équipe et la suivante. Tout va très vite sur les réseaux sociaux, et rester un certain temps sans activité peut se révélé très handicapant. Ensuite, certes, les comptes ont trouvé un certain écho sur la toile. Cependant, notre cible, les potentiels futurs élèves du département, ne sont pas encore atteint par notre communication.

\paragraph{}
Maintenant qu'une base pour notre communauté a été formée, il faut essayer de rendre la page active au maximum afin que celle-ci semble attirante pour tout potentiel futur élève y accédant la première fois. Il faut donner envie de venir et revenir.