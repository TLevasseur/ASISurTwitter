\section{Approche adoptée}

\paragraph{}
La première étape a donc été de créer les différents comptes. Sur Twitter, on peut nous trouver via le compte \symbol{64}ASI\_INSARouen, et via la page "Département ASI - INSA Rouen" sur Facebook. Dès leur création, nous avons demandé, par mail et à l'oral, à tous les étudiants du département de nous suivre sur ces deux plateformes, et nous les avons invité à nous partager toute information jugées intéressantes à partager.

\paragraph{}
Ensuite a commencé notre fouille du net à la recherche de contenu à suivre et partager. Nous avons donc, sur Twitter, suivi de nombreux comptes du monde de l'informatique, des études supérieures et de la région : les différentes écoles du groupe INSA, d'autres écoles telles que l'ENSIBS, les étudiants du département, la région Haute-Normandie, la ville de Saint-Etienne-du-Rouvray, Codeurs en Seine, ou encore CodinGame. Ces abonnements nous ont permis de retweeter certaines informations intéressantes. 

\paragraph{}
Nous sommes également resté à l'affût d'informations provenant de notre école. Les mails de la direction, de la secrétaire du département, de M. Gasso, ont été décortiqué et analysé par nos soins afin de ne pas louper un scoop à partager. Nous nous sommes rendus à la plupart des conférences, à la journée sur la sécurité des systèmes d'information, à Codeurs en Seine, muni de nore appareil photo pour ajouter un peu de visuel à notre page Facebook.

\paragraph{}
Outre ces informations, pour la plupart très sérieuses, nous avons décidé d'agrémenter nos comptes sur les réseaux sociaux de publications humoristiques, plus légères. Toujours orientées informatique ou département ASI, ces digressions permettent de changer le ton de notre propos et de rendre nos pages plus attrayantes, notamment pour un jeune public, qui nous intéresse particulièrement. Vidéos YouTube sur des avancées technologiques insolites, plaisanteries sur la vie de nos étudiants... Au-delà de ces publications, nous avons fait le choix de donner aux deux comptes, et particulièrement Twitter, un ton loin d'être formel.